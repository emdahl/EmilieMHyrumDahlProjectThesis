Sammenlikning av resultatene fra spektrometer og med SilCamLinsen. 
//
Klassifisering basert på PCA-resultatene










VISUAL LIGHT
Experiments analyzing the wave spectra of different types of plastics, have already been conducted. However, the results has been directed towards viewing the wave length interval describing near infrared light (NIR). Although the use of infrared light is an effective means of unobtrusive observation on land, it is far less effective in the ocean because long wavelength light is rapidly attenuated by seawater.

%https://oceanoptics.com/plastic-recycling-nir-spectroscopy/
Another important observation in these experiments, is the condition of the microplastic used. The material is pure and white, making the results independent of color. 

Water absorption 
%https://commons.wikimedia.org/wiki/File:Absorption_spectrum_of_liquid_water.png
This logarithmic (log-log) graph shows water’s absorption behavior at different colors wavelength. As seen in the graph, water absorption is minimised between 400 -600 nm


Light Transmission in the Ocean: http://www.waterencyclopedia.com/La-Mi/Light-Transmission-in-the-Ocean.html
https://manoa.hawaii.edu/exploringourfluidearth/physical/ocean-depths/light-ocean


AN IDEA:
In order to obtain a color-free spectrum, one could subtract the color spectrum from the resulting spectrum retrieved from the experiments. However, most pieces of plastic do not have one specific color, making it difficult to determine the spectrum to subtract. 

In addition. When experimenting with different types of plastics having more or less the same colors, the results turned out next to identical. At first, a thought was that this was due to the color being too dominant in relation to the rest of the properties. However, conducting the same experiment with non-colored pieces of plastic, gave the same end-result; no difference in spectral signature even if the types of plastics differ. - this could also be due to reflectance disturbance from "shiny" see through material. 

ANOTHER IDEA: 
What if we could inspect the level of additive acceptance in the different types of plastic. By finding the most resistant types, it might be possible to modify these into doing the same job as the types accepting additives. This way, we could reduce the production of the largest vectors/toxin-carriers. 