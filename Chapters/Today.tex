\todo[inline]{vurderer å ha hele/deler av seksjonen om plast fra motivation/background her}

\begin{comment}
Marine biological diversity is already exposed to climate change, overfishing and other man-made disruptions. As if this were not enough, plastic pollution also causes huge damage to the marine environment. (*)
\\
335 million metric tons of plastic was produced in 2016. %(https://www.statista.com/statistics/282732/global-production-of-plastics-since-1950/)
It is projected that the production will nearly double within the next 10-15 years
%https://wedocs.unep.org/bitstream/handle/20.500.11822/25398/WED%20Messaging%20Two-Page%2027April.pdf?sequence=12&isAllowed=y
\\\\

Kanehiro et al. (1995) states that plastic accounted for 80-85 percent of the seabed waste in Tokyo Bay in the 90's. (*)
This is a striking finding, considering that most plastic residues are floating to some degree. Different types of plastic have different densities. Some types have higher density than water and float, while other types are denser than water. This contributes to the fact that plastic can be found throughout the sea column, and is present in many ecosystems. For instance, plastics have been found in the digestive system of organisms of all sizes, from small marine invertebrates to whales.
\\\\
Polyethylene bags, operating in the ocean currents, have a large resemblance to the predators targeted by turtles. (*) Plastic debris can thus prevent their survival (Bugoni et al., 2001, Duguy et al., 1998). The ingestion of plastic debris can, among other things, reduce food intake, cause internal damage, strangling (hvordan bøyes dette) or even death after blockage of the intestinal tract. (Zitko and Hanlon, 1991).
\\\\
Plastic is also a potential carrier of chemicals and can absorb bacteria already present in the ocean, including polychlorinated biphenyls (PCBs). (*) PCBs are harmful chemicals, even in very low amounts. The ingestion of PCB can cause reproductive disorders, change the hormone levels and increase the risk of several diseases. In some cases, an intake may lead to death [(Ryan et al., 1988, Lee et al., 2001)]. Ryan et al. (1988) proved that PCB in the bird's tissue originates from plastic particles. Plastic pellets can thus be transporter for PCB in marine food chains (Mato et al., 2001).
\\\\
Furthermore, different types of bacteria are attracted to free floating marine debris. These are better known as “hitch hikers”, and can threaten sensitive coastal environments, as the bacteria are far from their natural habitats. 
%(Environmental implications of plastic debris in marine settings-entanglement, ingestion, smothering, hangers-on, hitch-hiking and alien invasions Murray R. Gregory)
\\\\
Some phyto-plankton eating species are particularly exposed, as microplastic easily can be confused with phyto-plankton. One of the most common types of microplastics, polystyrene (PS), has shown to affect the ability to reproduce. %(http://www.pnas.org/content/113/9/2430).
%* = The pollution of the marine environment by plastic debris: a review Jose G.B. Derraik *
\end{comment}

\section{Today}
Solving today’s plastic problem is not an easy task - especially when plastic and microplastic are not only present in the water surface, but in the entire water column. Imagine multiplying the entire ocean surface with a few hundred meters dept. This leaves an almost impossible starting point. In addition, there are continuous currents and motion, allowing a large spread. On top of it all, the deep sea is difficult to reach, and if reaching it, poor sight is often a fact. So, what have been done so far in order to solve this plastic problem?
\\\\
The problem with plastic is becoming an increasingly known problem. Since plastic contamination and many of its consequences are visible to the naked eye, few people can deny that the pollution of plastic is a fact. Nevertheless, it is not enough to be aware of the problem – the world must cooperate and act. Today there are solutions like ocean cleanup, etc\todo{mer av dette}. The challenge with these solutions is that they do not necessarily go towards high concentrated areas. Today we know that large pieces of plastic travel to the garbage patch ....\todo{og dette} (Evidence that the Great Pacific Garbage Patch is rapidly accumulating plastic 2018, L. Lebreton1,2, B. Slat1, F. Ferrari1 – mer herfra). Nevertheless, the entire ocean needs to be mapped.
\\\\
One obvious requirement when mapping and cleaning the ocean of plastic, is the availability of an instrument able to detect plastic in the ocean separating it from the rest of the ocean particles. This is where the field of spectroscopy enters the court. The development of image detectors, especially the two-dimensional silicon charge coupled device (CCD), has revolutionized image spectroscopy. CDD provides information on the distribution of photon intensity along the spectrograph's entrance slit. The distribution of entrance slit into different wavelengths and intensities, has made it possible to reconstruct detailed images at high spatial (defined as 1 cm) and spectral (defines as 1 nm) resolution. This makes a hyperspectral imager particularly suitable, as it consists of a spectrometer equipped with this charge coupled device (CCD). 
% (Development of hyperspectral imaging as a bio-optical taxonomic tool for pigmented marine organisms - geir)

%(BASIC HYPER SPECTRAL IMAGING F. SIGERNES)
%G. Vane, ed., Imaging Spectroscopy II, Proc. SPIE 834, 1988.
%W.L. Wolfe, Introduction to Imaging Spectrometers, Vol. TT25 of Tutorial Text Series, SPIE Press, Bellingham, Wash., pp. 1-147, 1997.
The last decade, most of the work and discovery in hyperspectral imaging, have been within space and air-born sensors. Here, the instrument has proven to be a particularly powerful remote control tool. It is not until recently that the technology has been tested underwater. As the sea is not stagnant and objects in water can behave differently than in air, new challenges can occur. Nevertheless, research shows that the technology is promising, also underwater, and can be a very useful tool when detecting microorganisms.
\\\\
%In addition, work on detection and classification of plastic in the sea, has been conducted, using hyperspectral imaging underwater. \todo{mer fra bert sitt arbeid}
%Bert: http://journals.sagepub.com/doi/pdf/10.1255/jnirs.1212 (mer herfra). 
Experiments analyzing the wave spectra of different types of plastics, have already been conducted. However, the results has been directed towards viewing the wave length interval describing near infrared light (NIR). Although the use of infrared light is an effective means when doing observation on land, light is rapidly attenuated by seawater which makes this approach far less effective when taking the problem below surface.

Therefore, experiments using hyperspectral imaging in the spectra of visual light is put on the agenda for this thesis. 

On a final note, the Silhouette camera (SilCam) developed by SINTEF, is also interesting in this matter. This technology is different, as it utilizes a light source behind the object to be identified to clearly see the outline of the target. Simplified, the hyperspectral imagery extracts the contents of what is depicted, while the SilCam captures the exterior. The question is, can these two methods compliment each other and give an even better way of detecting microorganisms? This has never been tried before..

%JEG ER VELDIG USIKKER PÅ OM DENNE BØR KOMME HER, SOM EN TEASER TIL NOE SOM EGENTLIG KOMMER I FURTHER WORK OG DERMED I NESTE RUNDE...



