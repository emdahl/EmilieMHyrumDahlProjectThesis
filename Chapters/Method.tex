\section{Hyperspectral Imaging} 
%evt bare spektrometeret - teste alle de ulike typene plast som vi har kjøpt fra tyskland 


\subsection{Purpose}
The purpose of the experiment is to get a better understanding of the properties of specific plastic types. By looking at how light is reflected from different types of plastic, it will be possible to compare the plastic types and hopefully detect some kind of pattern. Initially, the plastic pellets will be examined in a dry environment, before being placed in water later on. These results will be compared in order to see if experiments carried out in dry environment can be representative for plastic pellets in wet environment. 

\subsection{Hypotheses}
$\bullet$ Different types of plastic provide different signatures in the visible light spectrum\\ 
$\bullet$ Various types of plastic provide different signatures in infrared light spectrum\\
$\bullet$ Different types of plastic have different intensities of reflecting light\\
$\bullet$ Varying thickness in a specific type of plastic, will give different results\\
$\bullet$ Post consumer recycled pellets and clean pellets have different signatures\\
%Hvis dette stemmer, må vi finne ut om plasten noen gang kan anses som ”clean” i livsløpet (for eksempel på starten/etter hvert som farge osv er vasket av)
$\bullet$ The same type of plastic gives a similar signature in water as well as in a dry environment

\subsection{Material}
\begin{center}
\Rotatebox{0}{
\begin{tabular}{ |c|c|c|c|c| }
 \hline
 \textbf{Ref Number} & \textbf{State} & \textbf{Class} & \textbf{Additives} \\ 
 
 CRT131.00 & Post-Industrial Recyclate Pellets & PE: LDPE/LLDPE & Yes, colorants\\
 CRT150.00 & Post-Consumer Recyclate Regrind & PE-HD & Colorants \\
CRT170.00 & Environmental Pellets & PE: LDPE/HDPE & Yes \\
CRT171.00 & Environmental Fragments (Regrind) & PE-HD & Yes \\
CRT200.00 & Pristine Pellets & PP-Homopolymer & Yes, stabilizers \\
CRT250.00 & Post Consumer Recyclate: Pellets & PP Mixture & Yes \\
CRT300.00 & Pristine Pellets & PS General purpose & Unknown \\
CRT331.00 & Post Consumer Recyclate Regrind & PS Mixture & Yes \\
CRT400.00 & Pristine Pellets & PET Amorphous & No, not intentionally \\
CRT451.00 & Post Consumer Recyclate Regrind & PET Amorphous & No, not intentionally \\
CRT500.00 & Pellets & PVC Soft & Yes, softener \\
CRT530.00 & Pellets & PVC Hard & Yes, softener \\
 \hline
\end{tabular}
}
\end{center}
%$\bullet$ plastpose\\\\
%$\bullet$ plastflaske\\\\
%$\bullet$ …?

\subsection{Equipment} 
(from ocean optics web page)\\
$\bullet$ QE Pro spectrometer - measures optical signatures on the specific object. In this experiment, reflectance is used to illustrate the color properties\\
$\bullet$ WS-1 Reflectance Standards - 100 percent reflectance within 250-1500 nanometers\\
$\bullet$ QR400-7-VIS-BX reflection probe, light out and reflectance in\\
$\bullet$ RPH reflective probe holder, holding the reflection probe in a 45 degree angle and keeping the source 1 cm away from the measured object\\
$\bullet$ OceanView 1.6.7 - The program used to visualize the object reflectance \\
$\bullet$ Light source

\subsection{Procedure} 
The spectrometer is turned on and connected to OceanView 1.6.7 via an external computer. In OceanView, the spectrometer is set to a constant temperature of -10 degrees (Celsius). The QR400-7-VIS-BX reflection probe is connected to the spectrometer through one cable, and a light source through another. In addition, the reflection probe is placed in a reflective probe holder, angling the light output (and the sensor inlet) in order to avoid specular reflection. This angle is set to 45 degrees.
\\\\
Furthermore, the reflective probe is placed over the object to be measured. To begin with, this is the WS-1 Reflectance Standards with 100 percent reflectance within the range of 250-1500 nanometers. This is measured in OceanView and sets the standard for white. Furthermore, the reflective probe holder is placed over the specific plastic type to be measured. The reflectance from the plastic pieces is then visualized in OceanView and is ready for comparison. This process will be repeated with different types of plastic.
\\\\
%Underwater hyperspectral imaging: a new tool for marine archaeology 2018, ØYVIND ØDEGÅRD,1,2,* AKSEL ALSTAD MOGSTAD,3 GEIR JOHNSEN,3 ASGEIR J. SØRENSEN, AND MARTIN LUDVIGSEN1
We assume ideal lighting such that the measured radiance of an object is independent of its position on the scan line. Additionally, the reference plate is assumed to reflect the downwelling radiance equally at all wave- lengths. 

%Samme gjennomføres i vann?
%Samme gjennomføres med store plastbiter?

%Ønsket resultat?

\section{Underwater Hyperspectral Imaging} 

\section{Introducing a SilCam Lens} Implementere linsen i tidligere oppsett. 

\section{PCA} Denne baseres på dataen funnet fra metodene over: finner ut hvor sensitiv plasttypen er for forskjellige bølgelengder - er dette en plasttype som gir samme signatur over alle bølgelengder og er uniform feks?

\subsection{Classification} Herunder klassifisering og kategorisering - feks basert på signatur. (Leter etter en trender i datasettet.)

\section{Measurement of Success}
Målestokk for suksess